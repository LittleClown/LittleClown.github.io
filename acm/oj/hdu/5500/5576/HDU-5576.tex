\documentclass[a4paper]{ctexart}
\usepackage{graphicx,graphpap}
\usepackage{amsmath,amssymb}
\usepackage{indentfirst}
\usepackage{subfigure}
\usepackage{enumerate}
\usepackage{longtable}
\usepackage{pgfplots}
\usepackage{listings}
\usepackage{titlesec}
\usepackage{hyperref}
\usepackage{geometry}
\usepackage{fancyvrb}
\usepackage{fancyhdr}
\usepackage{tabularx}
\usepackage{Tabbing}
\usepackage{caption}
\usepackage{ifthen}
\usepackage{lineno}
\usepackage{xcolor}
\usepackage{array}
\usepackage{tikz}
\usepackage{calc}
\usepackage{cite}
\usepackage{CJK}
\usepackage{bm}

\geometry{left=2.5cm,right=2.5cm,top=2.5cm,bottom=2.5cm}
% \paperwidth  15 true cm
% \paperheight 08 true cm

\hypersetup{
  hyperindex=true,pdfstartview=FitH,
  bookmarksnumbered=true,bookmarksopen=true,
  citecolor=violet,urlcolor=violet,linkcolor=violet,
  colorlinks=true
}
\lstset{
  numbers=left,stepnumber=1,numberstyle=\scriptsize,numbersep=7pt,
  showstringspaces=false,extendedchars=false,escapeinside=``,tabsize=4,
  identifierstyle=\color{black},
  keywordstyle=\color{violet}\bfseries,
  commentstyle=\color{blue}small\bfseries
}

\begin{document}
\pagestyle{fancy}
\fancyhf{}
\fancyhead[C]{\heiti\color{violet}\bf Expection of String}
\fancyfoot[C]{\heiti\color{violet}\bf \thepage}
\fancyfoot[R]{\heiti\color{violet}\bf by Clown}

\section*{\heiti\color{violet}\bf\Huge \href{http://acm.hdu.edu.cn/showproblem.php?pid=5576}{Expection of String}}\par\vskip 3.5em
\subsection*{\heiti\bf\Large 问题简析}
为了计算期望,我们直接求出所有的可能。 \par
对于一个串~$s$,记这个串的计算结果为~$\varphi(s)$,并设串的长度为~$N$。 \par
因为,每一次交换后会得到多个可能的串,我们用~$\Gamma(k,p)$~表示执行~$k$~次交换后,所有乘号在位置~$p$~处的串构成的可重集合。
所以我们想要的答案就是~\quad $\displaystyle \sum_{p=1}^N \sum_{s\in\Gamma(K,p)} \varphi(s)$ \par\vskip 1em
设,$\displaystyle f(k,p,a,b)=\sum_{s\in\Gamma(k,p)} s_a\times s_b$。 \par\vskip 1em
注意到,
$\displaystyle \varphi(s)=\left(\sum_{a=1}^{p-1} s_a\times10^{p-a-1}\right)\times\left(\sum_{b=p+1}^N s_b\times10^{N-b}\right)%
=\sum_{a=1}^{p-1}\sum_{b=p+1}^N s_a\times s_b\times10^{p-a-1+N-b}$ \par\vskip 1em
所以, \quad 
$\displaystyle \sum_{p=1}^N\sum_{s\in\Gamma(K,p)} \varphi(s)=\sum_{p=1}^N\sum_{a=1}^{p-1}\sum_{b=p+1}^N f(K,p,a,b)\times10^{p-a-1+N-b}$ \par\vskip 1em
\noindent 接着,考虑如何进行~$f(k,p,a,b)$~的状态转移。 \par\vskip 1em
\newcounter{my:tab-counter}[table]
\setcounter{my:tab-counter}{0}
\begin{tabbing}
\hskip 2em \= \hskip 1.5em \= \hskip 1em \= \hskip 4em \= \hskip 1em \= \hskip 4em \= \hskip 1em \= \hskip 4em \kill
\> \stepcounter{my:tab-counter}\arabic{my:tab-counter}. \> $a$ \> 动 \> $b$ \> 动 \> $p$ \> 动 \\
\> \stepcounter{my:tab-counter}\arabic{my:tab-counter}. \> $a$ \> 动 \> $b$ \> 动 \> $p$ \> 不动 \\
\> \stepcounter{my:tab-counter}\arabic{my:tab-counter}. \> $a$ \> 动 \> $b$ \> 不动 \> $p$ \> 动 \\
\> \stepcounter{my:tab-counter}\arabic{my:tab-counter}. \> $a$ \> 动 \> $b$ \> 不动 \> $p$ \> 不动 \\
\> \stepcounter{my:tab-counter}\arabic{my:tab-counter}. \> $a$ \> 不动 \> $b$ \> 动 \> $p$ \> 动 \\
\> \stepcounter{my:tab-counter}\arabic{my:tab-counter}. \> $a$ \> 不动 \> $b$ \> 动 \> $p$ \> 不动 \\
\> \stepcounter{my:tab-counter}\arabic{my:tab-counter}. \> $a$ \> 不动 \> $b$ \> 不动 \> $p$ \> 动 \\
\> \stepcounter{my:tab-counter}\arabic{my:tab-counter}. \> $a$ \> 不动 \> $b$ \> 不动 \> $p$ \> 不动 \\
\end{tabbing} \par
\noindent 我们逐一进行考虑,则所有情况的和就是~$f(k+1,p,a,b)$。
\begin{enumerate}[\hskip 2em 1.]
\item 显然不可行,故贡献为: \quad 0
\item 即,$a$~上的字符和~$b$~上的字符交换,贡献为: \quad $f(k,p,b,a)$
\item 即,$a$~上的字符和~$p$~上的字符交换,贡献为: \quad $f(k,a,p,b)$
\item 即,$a$~上的字符和剩下的~$N-3$~个字符中任一字符交换,贡献为: \quad 
$\displaystyle \sum_{1\leqslant i\leqslant N,~i\neq a,~i\neq b,~i\neq p} f(k,p,i,b)$ \label{enum:4.}
\item 即,$b$~上的字符和~$p$~上的字符交换,贡献为: \quad $f(k,b,a,p)$
\item 即,$b$~上的字符和剩下的~$N-3$~个字符中任一字符交换,贡献为: \quad
$\displaystyle \sum_{1\leqslant i\leqslant N,~i\neq a,~i\neq b,~i\neq p} f(k,p,a,i)$ \label{enum:6.}
\item 即,$p$~上的字符和剩下的~$N-3$~个字符中任一字符交换,贡献为: \quad
$\displaystyle \sum_{1\leqslant i\leqslant N,~i\neq a,~i\neq b,~i\neq p} f(k,i,a,b)$ \label{enum:7.}
\item 即,在剩下的~$N-3$~个字符中任选两个字符,贡献为: \quad ${N-3 \choose 2}\times f(k,p,a,b)$
\end{enumerate} \par\vskip 2em
对于~{\heiti\bf \ref{enum:4.}}、 {\heiti\bf \ref{enum:6.}}、 {\heiti\bf \ref{enum:7.}}~利用前缀和,可以在~$O(1)$~的时间内计算出来。 \par\bigskip
算法总复杂度为 \quad $O(n^4)$。
\par\vskip 3em
\subsection*{\heiti\bf\Large 参考链接}
\href{http://async.icpc-camp.org/d/304-shanghai-2015-e-expection-of-string/4}
{\bf http://async.icpc-camp.org/d/304-shanghai-2015-e-expection-of-string/4}

\end{document}
